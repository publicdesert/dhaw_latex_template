\documentclass[a4paper,12pt]{report}

% Kodierung und Sprache
\usepackage[utf8]{inputenc}
\usepackage[T1]{fontenc}
\usepackage[scaled]{helvet}
\renewcommand*\familydefault{\sfdefault}
\usepackage[german]{babel}
\usepackage{graphicx}
\graphicspath{ {./images/} }
\usepackage{xcolor}
\usepackage{graphicx}
\usepackage{setspace}
\usepackage{titlesec}
\usepackage{fancyhdr}

% Für Blindtext. Kann entfernt werden.
\usepackage{lipsum}

% Titel, Autor, etc
\newcommand{\printAuthorName}{Max Mustermann}
\newcommand{\printAuthorMail}{max.mustermann@example.com}
\newcommand{\printDocumentTitle}{Titel meiner Studienarbeit}
\newcommand{\printDocumentSubject}{Mensch-Maschine-Interaktion}
\newcommand{\printCourseOfStudy}{Informatik}
\newcommand{\printDocumentDate}{\today}

% Farben definieren
\definecolor{dhaw_red}{RGB}{112, 18, 42}

% Seitenränder
\usepackage[
    includeheadfoot,
    top=1.25cm,
    bottom=1.25cm,
    left=2.5cm,
    right=2.5cm
]{geometry}

\usepackage[hidelinks]{hyperref}
\hypersetup{
    pdfauthor={\printAuthorName},
    pdftitle={\printDocumentTitle},
    pdfsubject={\printDocumentSubject},
}

% Literatur
\usepackage[backend=biber,style=apa,hyperref=true]{biblatex}
\addbibresource{literature/literature.bib}

% Kopf- und Fußzeile definieren
\pagestyle{fancy}
\fancyhf{} % alle Kopf-/Fußzeilen-Einträge leeren

\fancyhead[L]{\color{dhaw_red}{\textbf{\printDocumentTitle}}}
\fancyfoot[R]{\thepage}
\renewcommand{\headrulewidth}{0.4pt}
\renewcommand{\footrulewidth}{0pt}
\renewcommand{\headrule}{\color{dhaw_red}%
  \hrule width\headwidth height\headrulewidth
  \vskip-\headrulewidth
}

% Linien konfigurieren
\renewcommand{\headrulewidth}{0.4pt}
\renewcommand{\footrulewidth}{0pt}

% Genug Platz für die Kopfzeile
\setlength{\headheight}{14pt}
\setlength{\headsep}{10pt}

% Plain-Stil anpassen (für Kapitelanfang usw.)
\fancypagestyle{plain}{%
  \fancyhf{}
  \fancyhead[L]{\color{dhaw_red}{\textbf{\printDocumentTitle}}}
  \fancyfoot[R]{\thepage}
  \renewcommand{\headrulewidth}{0.4pt}
  \renewcommand{\footrulewidth}{0pt}
  \renewcommand{\headrule}{\color{dhaw_red}%
    \hrule width\headwidth height\headrulewidth
    \vskip-\headrulewidth
  }
}

\setstretch{1.5}

% Kapitelüberschriften anpassen
\titleformat{\chapter}[hang]
  {\normalfont\huge\bfseries}
  {\thechapter}
  {1em}
  {}

% Abstände vor und nach Kapiteln
\titlespacing*{\chapter}{0pt}{-10pt}{10pt}
\titlespacing*{\section}{0pt}{5pt}{5pt}

\linespread{1.213}

\begin{document}


% Titelseite
\begin{titlepage}
  \newgeometry{margin=0mm}
  \thispagestyle{empty} % keine Kopf-/Fußzeile
  \noindent
  \colorbox{dhaw_red}{
    \parbox[c][0.25\paperheight][c]{\paperwidth}{%
      \centering
      \vspace{1cm}
      \hspace*{-10.5cm}\includegraphics[height=3.17cm]{dhaw_logo.png}\\[2cm]
    }
  }

  \vspace{3cm}
  \noindent
  \hspace{2.5cm}
  \begin{minipage}{\dimexpr\paperwidth - 2.5cm - 2.5cm\relax}
    {\huge \textbf{\printDocumentTitle}}\\[1cm]
    \textbf{Studiengang:} \printCourseOfStudy\\
    \textbf{Modul:} \printDocumentSubject\\
    \rule{\textwidth}{1pt}\\
    \textbf{Name:} \printAuthorName\\
    \textbf{E-Mail-Adresse:} \printAuthorMail\\
    \textbf{Datum:} \printDocumentDate\\
  \end{minipage}
\end{titlepage}
\restoregeometry

% Wechsel zu römischen Seitenzahlen für äußere Seiten
\pagenumbering{Roman}

\tableofcontents

\chapter*{Abkürzungsverzeichnis}
\addcontentsline{toc}{chapter}{Abkürzungsverzeichnis}

\listoffigures
\addcontentsline{toc}{chapter}{Abbildungsverzeichnis}

\listoftables
\addcontentsline{toc}{chapter}{Tabellenverzeichnis}

\cleardoublepage
% Wechsel zu arabischen Seitenzahl für eigtl. Arbeit
\pagenumbering{arabic}

% Eigentlicher Text wird aus anderem Verzeichnis inkludiert. Hier wird eine Datei für alle Kapitel verwendet.
% Je nach Umfang der Arbeit können auch einzelne Dateien pro Kaptiel inkludiert werden.
\chapter{Einleitung}
\lipsum[1-4]

\chapter{Kaptitel 1}
\lipsum[1-2]
\section{Abschnitt 1}
\lipsum[1-2]
\section{Abschnitt 2}
\lipsum[1-2]
\section{Abschnitt 3}
\lipsum[1-2]

\chapter{Kaptitel 2}
\lipsum[1-2]
\section{Abschnitt 1}
\lipsum[1-2]
\section{Abschnitt 2}
\lipsum[1-2]
\section{Abschnitt 3}
\lipsum[1-2]

\chapter{Fazit und Ausblick}
\lipsum[1-4]

\cleardoublepage
% Wechsel zu römischen Zahlen für restliche äußere Seiten
\pagenumbering{Roman}
\setcounter{page}{5} % Counter muss hier manuell angepasst werden

\printbibliography
\addcontentsline{toc}{chapter}{Literatur}

\chapter*{Anhang}
\addcontentsline{toc}{chapter}{Anhang}
\include{content/attachments.tex}

\end{document}
